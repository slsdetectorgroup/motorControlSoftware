\section{SLIT Class Reference}
\label{classSLIT}\index{SLIT@{SLIT}}
Defines the slit object parameters and some methods to do before calling a slit motor move function.  


{\tt \#include $<$SLIT.h$>$}

\subsection*{Public Member Functions}
\begin{CompactItemize}
\item 
\bf{SLIT} (double \bf{x1pos}, double \bf{x2pos})
\item 
double \bf{get\-Limit} ()
\item 
double \bf{get\-X1Limit} ()
\item 
double \bf{get\-X2Limit} ()
\item 
void \bf{set\-X1pos} (double new\-Position)
\item 
void \bf{set\-X2pos} (double new\-Position)
\item 
void \bf{set\-Bothpos} (double new\-Position1, double new\-Position2)
\item 
int \bf{can\-X1Move} (double new\-Position)
\item 
int \bf{can\-X2Move} (double new\-Position)
\item 
int \bf{can\-Both\-Slits\-Move} (double new\-Position1, double new\-Position2)
\item 
double \bf{get\-Slit\-Width} ()
\item 
double \bf{get\-X1Center} ()
\item 
void \bf{print} ()
\end{CompactItemize}
\subsection*{Private Attributes}
\begin{CompactItemize}
\item 
double \bf{x1pos}
\item 
double \bf{x2pos}
\item 
double \bf{x1Limit}
\item 
double \bf{x2Limit}
\item 
double \bf{Limit}
\end{CompactItemize}


\subsection{Detailed Description}
Defines the slit object parameters and some methods to do before calling a slit motor move function. 



\subsection{Constructor \& Destructor Documentation}
\index{SLIT@{SLIT}!SLIT@{SLIT}}
\index{SLIT@{SLIT}!SLIT@{SLIT}}
\subsubsection{\setlength{\rightskip}{0pt plus 5cm}SLIT::SLIT (double {\em x1pos}, double {\em x2pos})}\label{classSLIT_948e48d9410bb54fb949117d042a30b1}


Constructor creates the slit object with the positions of both the slits \begin{Desc}
\item[Parameters:]
\begin{description}
\item[{\em x1pos}]position of slit\_\-x1 \item[{\em x2pos}]position of slit\_\-x2 \end{description}
\end{Desc}


\subsection{Member Function Documentation}
\index{SLIT@{SLIT}!canBothSlitsMove@{canBothSlitsMove}}
\index{canBothSlitsMove@{canBothSlitsMove}!SLIT@{SLIT}}
\subsubsection{\setlength{\rightskip}{0pt plus 5cm}int SLIT::can\-Both\-Slits\-Move (double {\em new\-Position1}, double {\em new\-Position2})}\label{classSLIT_c87e899d21cc8bf0abf626e18aab37fd}


returns 0 if both slits can move simultaneously to new position without crash, else 1 or -1 \begin{Desc}
\item[Returns:]0 if both slits can move simultaneously to new position without crash, else 1 or -1 \end{Desc}
\index{SLIT@{SLIT}!canX1Move@{canX1Move}}
\index{canX1Move@{canX1Move}!SLIT@{SLIT}}
\subsubsection{\setlength{\rightskip}{0pt plus 5cm}int SLIT::can\-X1Move (double {\em new\-Position})}\label{classSLIT_ff3c5b32c92f6fa9f5eafe67f74173c1}


returns 0 if slit\_\-x1 can move to new position without crash, else 1 or -1 \begin{Desc}
\item[Returns:]0 if slit\_\-x1 can move to new position without crash, else 1 or -1 \end{Desc}
\index{SLIT@{SLIT}!canX2Move@{canX2Move}}
\index{canX2Move@{canX2Move}!SLIT@{SLIT}}
\subsubsection{\setlength{\rightskip}{0pt plus 5cm}int SLIT::can\-X2Move (double {\em new\-Position})}\label{classSLIT_44e0b9c5bef453856da21df389c4d7b1}


returns 0 if slit\_\-x2 can move to new position without crash, else 1 or -1 \begin{Desc}
\item[Returns:]0 if slit\_\-x2 can move to new position without crash, else 1 or -1 \end{Desc}
\index{SLIT@{SLIT}!getLimit@{getLimit}}
\index{getLimit@{getLimit}!SLIT@{SLIT}}
\subsubsection{\setlength{\rightskip}{0pt plus 5cm}double SLIT::get\-Limit ()}\label{classSLIT_d2a021371c2ab19c84683a6613a60ff4}


returns the Limit member for slit object \begin{Desc}
\item[Returns:]the Limit member for slit object \end{Desc}
\index{SLIT@{SLIT}!getSlitWidth@{getSlitWidth}}
\index{getSlitWidth@{getSlitWidth}!SLIT@{SLIT}}
\subsubsection{\setlength{\rightskip}{0pt plus 5cm}double SLIT::get\-Slit\-Width ()}\label{classSLIT_a820571f1254ac87a18c6aed03495b4c}


gets the distance between the slit motors \begin{Desc}
\item[Returns:]the width between the slit motors \end{Desc}
\index{SLIT@{SLIT}!getX1Center@{getX1Center}}
\index{getX1Center@{getX1Center}!SLIT@{SLIT}}
\subsubsection{\setlength{\rightskip}{0pt plus 5cm}double SLIT::get\-X1Center ()}\label{classSLIT_535c0fe821232b915d82c19e387bfaf1}


gets the center of the slits compared to x1 measurements \begin{Desc}
\item[Returns:]the center of the slits compared to x1 measurements \end{Desc}
\index{SLIT@{SLIT}!getX1Limit@{getX1Limit}}
\index{getX1Limit@{getX1Limit}!SLIT@{SLIT}}
\subsubsection{\setlength{\rightskip}{0pt plus 5cm}double SLIT::get\-X1Limit ()}\label{classSLIT_e5831b97d54a6ea447a92aeb4899a451}


returns the limit position of slit\_\-x1 \begin{Desc}
\item[Returns:]the limit position of slit\_\-x1 \end{Desc}
\index{SLIT@{SLIT}!getX2Limit@{getX2Limit}}
\index{getX2Limit@{getX2Limit}!SLIT@{SLIT}}
\subsubsection{\setlength{\rightskip}{0pt plus 5cm}double SLIT::get\-X2Limit ()}\label{classSLIT_5a161e955e7c0b2f63b5f053ac00dd76}


returns the limit position of slit\_\-x2 \begin{Desc}
\item[Returns:]the limit position of slit\_\-x2 \end{Desc}
\index{SLIT@{SLIT}!print@{print}}
\index{print@{print}!SLIT@{SLIT}}
\subsubsection{\setlength{\rightskip}{0pt plus 5cm}void SLIT::print ()}\label{classSLIT_89fed1e0eb4f61270d4c86643f296958}


for debugging: prints all the members of the slit object \index{SLIT@{SLIT}!setBothpos@{setBothpos}}
\index{setBothpos@{setBothpos}!SLIT@{SLIT}}
\subsubsection{\setlength{\rightskip}{0pt plus 5cm}void SLIT::set\-Bothpos (double {\em new\-Position1}, double {\em new\-Position2})}\label{classSLIT_ca8deba292e1823cfee90f5c1be493a0}


sets the positions and the limits of both the slits \index{SLIT@{SLIT}!setX1pos@{setX1pos}}
\index{setX1pos@{setX1pos}!SLIT@{SLIT}}
\subsubsection{\setlength{\rightskip}{0pt plus 5cm}void SLIT::set\-X1pos (double {\em new\-Position})}\label{classSLIT_0f2d600483fc0711d342324157591689}


sets the new slit\_\-x1 position and its limit \index{SLIT@{SLIT}!setX2pos@{setX2pos}}
\index{setX2pos@{setX2pos}!SLIT@{SLIT}}
\subsubsection{\setlength{\rightskip}{0pt plus 5cm}void SLIT::set\-X2pos (double {\em new\-Position})}\label{classSLIT_f87749bc634265a89c282d8c596cae8d}


sets the new slit\_\-x2 position and its limit 

\subsection{Member Data Documentation}
\index{SLIT@{SLIT}!Limit@{Limit}}
\index{Limit@{Limit}!SLIT@{SLIT}}
\subsubsection{\setlength{\rightskip}{0pt plus 5cm}double \bf{SLIT::Limit}\hspace{0.3cm}{\tt  [private]}}\label{classSLIT_a19aa07ec71d7247e6eeca8b4eb1b098}


Limiting value \index{SLIT@{SLIT}!x1Limit@{x1Limit}}
\index{x1Limit@{x1Limit}!SLIT@{SLIT}}
\subsubsection{\setlength{\rightskip}{0pt plus 5cm}double \bf{SLIT::x1Limit}\hspace{0.3cm}{\tt  [private]}}\label{classSLIT_3bc3810f3692716f10408706d2f35ac4}


Slit\_\-x1 limit position \index{SLIT@{SLIT}!x1pos@{x1pos}}
\index{x1pos@{x1pos}!SLIT@{SLIT}}
\subsubsection{\setlength{\rightskip}{0pt plus 5cm}double \bf{SLIT::x1pos}\hspace{0.3cm}{\tt  [private]}}\label{classSLIT_3f05f690fd412f0a36cbe96310dbcb35}


Slit\_\-x1 position \index{SLIT@{SLIT}!x2Limit@{x2Limit}}
\index{x2Limit@{x2Limit}!SLIT@{SLIT}}
\subsubsection{\setlength{\rightskip}{0pt plus 5cm}double \bf{SLIT::x2Limit}\hspace{0.3cm}{\tt  [private]}}\label{classSLIT_69099775d9c8a802a84bd0d0a0fca1fa}


Slit\_\-x2 limit position \index{SLIT@{SLIT}!x2pos@{x2pos}}
\index{x2pos@{x2pos}!SLIT@{SLIT}}
\subsubsection{\setlength{\rightskip}{0pt plus 5cm}double \bf{SLIT::x2pos}\hspace{0.3cm}{\tt  [private]}}\label{classSLIT_8fd69fb384309c14071f088586c902b1}


Slit\_\-x2 position 

The documentation for this class was generated from the following files:\begin{CompactItemize}
\item 
XRay\-Box\-Server/\bf{SLIT.h}\item 
XRay\-Box\-Server/\bf{SLIT.cpp}\end{CompactItemize}
